\definecolor{ColA}{HTML}{991F3D}
\definecolor{ColB}{HTML}{997A1F}
\definecolor{ColC}{HTML}{3D991F}
\definecolor{ColD}{HTML}{1F997A}
\definecolor{ColE}{HTML}{1F3D99}
\definecolor{ColF}{HTML}{7A1F99}
\section{Simulations}

We will estimate the echo-return loss enhancement (ERLE) of the estimated filter $\bvh[h]'[k]$ from \cref{eq:sec2:def_bvhh'_k}, for distinct scenarios. We generated results for different numbers of crossband filters $K$, comparing them between the STFT and SSBT. Continuous lines are the results with the SSBT, and dashed lines with the STFT. Each color represents a different $K$.

\noindent\textit{Case 1 -} In here: both $x[n]$ and $v[n]$ are white Gaussian noises; $L = \floor{\nicefrac{N}{2}}$; we use a Hann window; and $N = 32$ samples per window.

\begin{figure}[H]
	\centering
	%% Requires:
% pgfplots.sty
% edit_pgfplots.tex

\pgfplotsset{compat=1.18}
%\begin{subfigure}{\linewidth}
%\centering
\tikzsetnextfilename{lineplot_erle_case1}
\begin{tikzpicture}
	\begin{lineplot}{ERLE (dB)}%[ymin=-0.1, ymax=3]
		\addplot [style=resA]
		table [col sep=comma, y=val] {figures/io_input/case1/ERLE__ssbt__K_0.csv};
		\addplot [style=resB]
		table [col sep=comma, y=val] {figures/io_input/case1/ERLE__ssbt__K_1.csv};
		\addplot [style=resC]
		table [col sep=comma, y=val] {figures/io_input/case1/ERLE__ssbt__K_2.csv};
		\addplot [style=resD]
		table [col sep=comma, y=val] {figures/io_input/case1/ERLE__ssbt__K_3.csv};
		\addplot [style=resE]
		table [col sep=comma, y=val] {figures/io_input/case1/ERLE__ssbt__K_4.csv};
		
		\addplot [style=resG]
		table [col sep=comma, y=val] {figures/io_input/case1/ERLE__stft__K_0.csv};
		\addplot [style=resH]
		table [col sep=comma, y=val] {figures/io_input/case1/ERLE__stft__K_1.csv};
		\addplot [style=resI]
		table [col sep=comma, y=val] {figures/io_input/case1/ERLE__stft__K_2.csv};
		\addplot [style=resJ]
		table [col sep=comma, y=val] {figures/io_input/case1/ERLE__stft__K_3.csv};
		\addplot [style=resK]
		table [col sep=comma, y=val] {figures/io_input/case1/ERLE__stft__K_4.csv};
		
		\addlegendentry{K=0, SSBT};
		\addlegendentry{K=1, SSBT};
		\addlegendentry{K=2, SSBT};
		\addlegendentry{K=3, SSBT};
		\addlegendentry{K=4, SSBT};
		
		\addlegendentry{K=0, STFT};
		\addlegendentry{K=1, STFT};
		\addlegendentry{K=2, STFT};
		\addlegendentry{K=3, STFT};
		\addlegendentry{K=4, STFT};
	\end{lineplot}
\end{tikzpicture}
%	\caption{}
%	\label{subfig:1_gain_lineplot}
%\end{subfigure}
	\caption{ERLE lineplot.}
	\label{fig:sec4:lineplot_erle_case1}
\end{figure}

%With the STFT, we see that although using $K=1$ is better than using $K=0$ for higher SNRs, we also see that increasing $K$ further doesn't lead to a better output, no matter the SNR. This is different than what was achieved with the SSBT, where increasing $K$ leads to a better performance for higher SNRs. This result for the STFT is also different than that obtained in Avargel's paper, where increasing $K$ led to a better performance for higher SNRs (fig. 9). This could be due to the RIR's used.
%
%We see that the SSBT has a better performance than the STFT for lower input SNRs for all $K \neq 0$, which outperformed all results obtained through the SSBT. For each $K$, the marker indicate the SNR value in which it is advantageous to use the STFT for higher SNRs, and the SSBT for lower ones.

\noindent\textit{Case 2 -} Now $x[n]$ is a male speech signal, and $v[n]$ is still a white Gaussian noise.

\begin{figure}[H]
	\centering
	%% Requires:
% pgfplots.sty
% edit_pgfplots.tex

\pgfplotsset{compat=1.18}
%\begin{subfigure}{\linewidth}
%\centering
\tikzsetnextfilename{lineplot_erle_case2}
\begin{tikzpicture}
	\begin{lineplot}{ERLE (dB)}%[ymin=-0.1, ymax=3]
		\addplot [style=resA]
		table [col sep=comma, y=val] {figures/io_input/case2/ERLE__ssbt__K_0.csv};
		\addplot [style=resB]
		table [col sep=comma, y=val] {figures/io_input/case2/ERLE__ssbt__K_1.csv};
		\addplot [style=resC]
		table [col sep=comma, y=val] {figures/io_input/case2/ERLE__ssbt__K_2.csv};
		\addplot [style=resD]
		table [col sep=comma, y=val] {figures/io_input/case2/ERLE__ssbt__K_3.csv};
		\addplot [style=resE]
		table [col sep=comma, y=val] {figures/io_input/case2/ERLE__ssbt__K_4.csv};
		
		\addplot [style=resG]
		table [col sep=comma, y=val] {figures/io_input/case2/ERLE__stft__K_0.csv};
		\addplot [style=resH]
		table [col sep=comma, y=val] {figures/io_input/case2/ERLE__stft__K_1.csv};
		\addplot [style=resI]
		table [col sep=comma, y=val] {figures/io_input/case2/ERLE__stft__K_2.csv};
		\addplot [style=resJ]
		table [col sep=comma, y=val] {figures/io_input/case2/ERLE__stft__K_3.csv};
		\addplot [style=resK]
		table [col sep=comma, y=val] {figures/io_input/case2/ERLE__stft__K_4.csv};
		
		\addlegendentry{K=0, SSBT};
		\addlegendentry{K=1, SSBT};
		\addlegendentry{K=2, SSBT};
		\addlegendentry{K=3, SSBT};
		\addlegendentry{K=4, SSBT};
		
		\addlegendentry{K=0, STFT};
		\addlegendentry{K=1, STFT};
		\addlegendentry{K=2, STFT};
		\addlegendentry{K=3, STFT};
		\addlegendentry{K=4, STFT};
	\end{lineplot}
\end{tikzpicture}
%	\caption{}
%	\label{subfig:1_gain_lineplot}
%\end{subfigure}
	\caption{ERLE lineplot.}
	\label{fig:sec4:lineplot_erle_case2}
\end{figure}

\noindent\textit{Case 3 -} In this scenario, $x[n]$ is a white Gaussian noise, and $v[n]$ is a male speech signal.

\begin{figure}[H]
	\centering
	%% Requires:
% pgfplots.sty
% edit_pgfplots.tex

\pgfplotsset{compat=1.18}
%\begin{subfigure}{\linewidth}
%\centering
\tikzsetnextfilename{lineplot_erle_case3}
\begin{tikzpicture}
	\begin{lineplot}{ERLE (dB)}%[ymin=-0.1, ymax=3]
		\addplot [style=resA]
		table [col sep=comma, y=val] {figures/io_input/case3/ERLE__ssbt__K_0.csv};
		\addplot [style=resB]
		table [col sep=comma, y=val] {figures/io_input/case3/ERLE__ssbt__K_1.csv};
		\addplot [style=resC]
		table [col sep=comma, y=val] {figures/io_input/case3/ERLE__ssbt__K_2.csv};
		\addplot [style=resD]
		table [col sep=comma, y=val] {figures/io_input/case3/ERLE__ssbt__K_3.csv};
		\addplot [style=resE]
		table [col sep=comma, y=val] {figures/io_input/case3/ERLE__ssbt__K_4.csv};
		
		\addplot [style=resG]
		table [col sep=comma, y=val] {figures/io_input/case3/ERLE__stft__K_0.csv};
		\addplot [style=resH]
		table [col sep=comma, y=val] {figures/io_input/case3/ERLE__stft__K_1.csv};
		\addplot [style=resI]
		table [col sep=comma, y=val] {figures/io_input/case3/ERLE__stft__K_2.csv};
		\addplot [style=resJ]
		table [col sep=comma, y=val] {figures/io_input/case3/ERLE__stft__K_3.csv};
		\addplot [style=resK]
		table [col sep=comma, y=val] {figures/io_input/case3/ERLE__stft__K_4.csv};
		
		\addlegendentry{K=0, SSBT};
		\addlegendentry{K=1, SSBT};
		\addlegendentry{K=2, SSBT};
		\addlegendentry{K=3, SSBT};
		\addlegendentry{K=4, SSBT};
		
		\addlegendentry{K=0, STFT};
		\addlegendentry{K=1, STFT};
		\addlegendentry{K=2, STFT};
		\addlegendentry{K=3, STFT};
		\addlegendentry{K=4, STFT};
	\end{lineplot}
\end{tikzpicture}
%	\caption{}
%	\label{subfig:1_gain_lineplot}
%\end{subfigure}
	\caption{ERLE lineplot.}
	\label{fig:sec4:lineplot_erle_case3}
\end{figure}

\noindent\textit{Case 4 -} Here, both signals are white Gaussian noise, but now $L = \floor{\nicefrac{N}{4}}$.

\begin{figure}[H]
	\centering
	%% Requires:
% pgfplots.sty
% edit_pgfplots.tex

\pgfplotsset{compat=1.18}
%\begin{subfigure}{\linewidth}
%\centering
\tikzsetnextfilename{lineplot_erle_case4}
\begin{tikzpicture}
	\begin{lineplot}{ERLE (dB)}%[ymin=-0.1, ymax=3]
		\addplot [style=resA]
		table [col sep=comma, y=val] {figures/io_input/case4/ERLE__ssbt__K_0.csv};
		\addplot [style=resB]
		table [col sep=comma, y=val] {figures/io_input/case4/ERLE__ssbt__K_1.csv};
		\addplot [style=resC]
		table [col sep=comma, y=val] {figures/io_input/case4/ERLE__ssbt__K_2.csv};
		\addplot [style=resD]
		table [col sep=comma, y=val] {figures/io_input/case4/ERLE__ssbt__K_3.csv};
		\addplot [style=resE]
		table [col sep=comma, y=val] {figures/io_input/case4/ERLE__ssbt__K_4.csv};
		
		\addplot [style=resG]
		table [col sep=comma, y=val] {figures/io_input/case4/ERLE__stft__K_0.csv};
		\addplot [style=resH]
		table [col sep=comma, y=val] {figures/io_input/case4/ERLE__stft__K_1.csv};
		\addplot [style=resI]
		table [col sep=comma, y=val] {figures/io_input/case4/ERLE__stft__K_2.csv};
		\addplot [style=resJ]
		table [col sep=comma, y=val] {figures/io_input/case4/ERLE__stft__K_3.csv};
		\addplot [style=resK]
		table [col sep=comma, y=val] {figures/io_input/case4/ERLE__stft__K_4.csv};
		
		\addlegendentry{K=0, SSBT};
		\addlegendentry{K=1, SSBT};
		\addlegendentry{K=2, SSBT};
		\addlegendentry{K=3, SSBT};
		\addlegendentry{K=4, SSBT};
		
		\addlegendentry{K=0, STFT};
		\addlegendentry{K=1, STFT};
		\addlegendentry{K=2, STFT};
		\addlegendentry{K=3, STFT};
		\addlegendentry{K=4, STFT};
	\end{lineplot}
\end{tikzpicture}
%	\caption{}
%	\label{subfig:1_gain_lineplot}
%\end{subfigure}
	\caption{ERLE lineplot.}
	\label{fig:sec4:lineplot_erle_case4}
\end{figure}

\noindent\textit{Case 5 -} We go back to $L = \floor{\nicefrac{N}{2}}$, but now using a Hamming window instead of a Hann one.

\begin{figure}[H]
	\centering
	%% Requires:
% pgfplots.sty
% edit_pgfplots.tex

\pgfplotsset{compat=1.18}
%\begin{subfigure}{\linewidth}
%\centering
\tikzsetnextfilename{lineplot_erle_case5}
\begin{tikzpicture}
	\begin{lineplot}{ERLE (dB)}%[ymin=-0.1, ymax=3]
		\addplot [style=resA]
		table [col sep=comma, y=val] {figures/io_input/case5/ERLE__ssbt__K_0.csv};
		\addplot [style=resB]
		table [col sep=comma, y=val] {figures/io_input/case5/ERLE__ssbt__K_1.csv};
		\addplot [style=resC]
		table [col sep=comma, y=val] {figures/io_input/case5/ERLE__ssbt__K_2.csv};
		\addplot [style=resD]
		table [col sep=comma, y=val] {figures/io_input/case5/ERLE__ssbt__K_3.csv};
		\addplot [style=resE]
		table [col sep=comma, y=val] {figures/io_input/case5/ERLE__ssbt__K_4.csv};
		
		\addplot [style=resG]
		table [col sep=comma, y=val] {figures/io_input/case5/ERLE__stft__K_0.csv};
		\addplot [style=resH]
		table [col sep=comma, y=val] {figures/io_input/case5/ERLE__stft__K_1.csv};
		\addplot [style=resI]
		table [col sep=comma, y=val] {figures/io_input/case5/ERLE__stft__K_2.csv};
		\addplot [style=resJ]
		table [col sep=comma, y=val] {figures/io_input/case5/ERLE__stft__K_3.csv};
		\addplot [style=resK]
		table [col sep=comma, y=val] {figures/io_input/case5/ERLE__stft__K_4.csv};
		
		\addlegendentry{K=0, SSBT};
		\addlegendentry{K=1, SSBT};
		\addlegendentry{K=2, SSBT};
		\addlegendentry{K=3, SSBT};
		\addlegendentry{K=4, SSBT};
		
		\addlegendentry{K=0, STFT};
		\addlegendentry{K=1, STFT};
		\addlegendentry{K=2, STFT};
		\addlegendentry{K=3, STFT};
		\addlegendentry{K=4, STFT};
	\end{lineplot}
\end{tikzpicture}
%	\caption{}
%	\label{subfig:1_gain_lineplot}
%\end{subfigure}
	\caption{ERLE lineplot.}
	\label{fig:sec4:lineplot_erle_case5}
\end{figure}

\noindent\textit{Case 6 -} We now use a Hann window with $N = 64$ samples.

\begin{figure}[H]
\centering
%% Requires:
% pgfplots.sty
% edit_pgfplots.tex

\pgfplotsset{compat=1.18}
%\begin{subfigure}{\linewidth}
%\centering
\tikzsetnextfilename{lineplot_erle_case6}
\begin{tikzpicture}
	\begin{lineplot}{ERLE (dB)}%[ymin=-0.1, ymax=3]
		\addplot [style=resA]
		table [col sep=comma, y=val] {figures/io_input/case6/ERLE__ssbt__K_0.csv};
		\addplot [style=resB]
		table [col sep=comma, y=val] {figures/io_input/case6/ERLE__ssbt__K_1.csv};
		\addplot [style=resC]
		table [col sep=comma, y=val] {figures/io_input/case6/ERLE__ssbt__K_2.csv};
		\addplot [style=resD]
		table [col sep=comma, y=val] {figures/io_input/case6/ERLE__ssbt__K_3.csv};
		\addplot [style=resE]
		table [col sep=comma, y=val] {figures/io_input/case6/ERLE__ssbt__K_4.csv};
		
		\addplot [style=resG]
		table [col sep=comma, y=val] {figures/io_input/case6/ERLE__stft__K_0.csv};
		\addplot [style=resH]
		table [col sep=comma, y=val] {figures/io_input/case6/ERLE__stft__K_1.csv};
		\addplot [style=resI]
		table [col sep=comma, y=val] {figures/io_input/case6/ERLE__stft__K_2.csv};
		\addplot [style=resJ]
		table [col sep=comma, y=val] {figures/io_input/case6/ERLE__stft__K_3.csv};
		\addplot [style=resK]
		table [col sep=comma, y=val] {figures/io_input/case6/ERLE__stft__K_4.csv};
		
		\addlegendentry{K=0, SSBT};
		\addlegendentry{K=1, SSBT};
		\addlegendentry{K=2, SSBT};
		\addlegendentry{K=3, SSBT};
		\addlegendentry{K=4, SSBT};
		
		\addlegendentry{K=0, STFT};
		\addlegendentry{K=1, STFT};
		\addlegendentry{K=2, STFT};
		\addlegendentry{K=3, STFT};
		\addlegendentry{K=4, STFT};
	\end{lineplot}
\end{tikzpicture}
%	\caption{}
%	\label{subfig:1_gain_lineplot}
%\end{subfigure}
\caption{ERLE lineplot.}
\label{fig:sec4:lineplot_erle_case6}
\end{figure}

We see that in none of the proposed cases, the SSBT results outperformed the STFT ones. The only case in which they were matched was case 2, with $x[n]$ being a speech signal, but in that scenario the ERLE was of $-2.2\text{dB}$ or lower, therefore incorrectly modeling $d[n]$.