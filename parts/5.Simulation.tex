\input{figures/io_input/colors_6.tex}
\section{Simulations}

% TODO: Comment on the results obtained and compare them among themselves.

In the simulations\footnote{The code for the simulations can be found in \url{https://github.com/VCurtarelli/py-ssb-ctf-bf}.}, we will use a sampling frequency $f_s = 16\si{\kilo\hertz}$. The sensor array is an uniform linear array of 10 sensors, with an intersensor distance of $2\si{\cm}$. All RIR's were generated using Habets' code. Signals used were from the SMARD and LINSE databases.

The room's dimensions are $4\m\times6\m\times3\m$, with a reverberation time of $0.11\si{\second}$. The desired source is located at $(2\m,~1\m,~1\m)$, and its signal is a male voice (SMARD, \texttt{50\_male\_speech\_english\_ch8\_OmniPower4296.flac}).

The interfering source is located (simultaneously) at $(0.5\m,~5\m,~[0.3:0.3:2.7]\m)$, simulating an open door, with its signal being a babble sound (LINSE database, \texttt{babble.mat}). The noise signal is a WGN noise (SMARD database, \texttt{wgn\_48kHz\_ch8\_OmniPower4296.flac}). All signals were resampled to the desired sampling frequency. The sensor array is assumed to be an uniform linear array, positioned at $(2\m,~[4.02:0.02:4.2]\m,~1\m)$. Its sensors are assumed to be omnidirectional and with flat frequency response.

We have that the input SNR between desired and interfering signals is of $0\dB$, and the SNR between desired and noise signals is of $30\dB$. The filters were calculated every 25 windows, and consider (up to) the previous 25 windows, in order to calculate the correlation matrices.

We will compare the filters obtained through the STFT and SSBT transforms, with T-SSBT denoting the beamformer obtained via the true-distortionless MVDR derived in \cref{sec4:true_distortionless_mvdr_ssbt}, and N-SSBT the naive approach, in which one would simply use \cref{eq:sec3:mvdr_beamformer} to calculate the SSBT beamformer. To assess their performances, the filter was brought back to the time domain, and transformed into the STFT domain, as this is the most mathematically faithful representation of the real situation. The analysis (and synthesis) windows have 32 samples, for all methods. In all lineplots, the STFT is presented in red, the N-SSBT in green, and the T-SSBT in blue.

\subsection{Results}

In \cref{fig:lineplot_dsdi_32} we have the DSDI for all three methods, window-averaged. \Cref{fig:lineplot_gain_32} shows the window-wise averaged gain in SNR, and \cref{fig:heatmap_gain_32} the gain in SNR for each window (with the windows being represented by the time started, in seconds), for all methods.

Although it isn't as clear from the results per window in \cref{fig:heatmap_gain_32}, \cref{fig:lineplot_gain_32} shows us that both beamformers obtained from the SSBT led to a better enhancement of the signal, in terms of the output SNR, with the N-SSBT beamformer having a better performance over (almost) all spectrum, and the T-SSBT beamformer being better for lower frequencies, tying with the STFT for higher ones.

Also, \cref{fig:lineplot_dsdi_32} shows that, much like the STFT beamformer, the T-SSBT filter also was able to ensure a distortionless response for the desired signal, a feature that wasn't achieved by the SSBT beamformer. This is wholly expected, since the N-SSBT beamformer was naively designed with the MVDR in mind, and wasn't fully planned to achieve a distortionless behavior in the STFT (and therefore the time) domain, while the T-SSBT took this into account on its derivation.

\input{figures/io_input/aux_data_32.tex}

\begin{figure}[H]
	\centering
	\input{figures/32.lineplot_dsdi.tex}
	\caption{Window-average DSDI.}
	\label{fig:lineplot_dsdi_32}
\end{figure}

\begin{figure}[H]
	\centering
	\input{figures/32.lineplot_gain.tex}
	\caption{Window-average SNR gain.}
	\label{fig:lineplot_gain_32}
\end{figure}

\begin{figure}[H]
	\centering
	\input{figures/32.heatmap_gain.tex}
	
	\tikzsetnextfilename{heatmap_gain_32}%
	\vspace*{0.4em}
	\ref*{heatmap_SSBT_32}
	\caption{Per-window SNR gain.}
	\label{fig:heatmap_gain_32}
\end{figure}
%
\subsection{Results - 64 samples/window}

In this simulation, all constants were maintained from the previous test, but now we use only 64 samples per window in the transforms, instead of the previous 32.

In here, from \cref{fig:lineplot_gain_64,fig:heatmap_gain_64} we see a similar result to that which was obtained previously, with the N-SSBT beamformer having a better performance overall, but causing some distortion in the desired signal; while the T-SSBT beamformer has a slightly better performance than the one obtained through the STFT, while also having a distortionless behavior, as is seen in \cref{fig:lineplot_dsdi_64}.

%We see that the filters for both transforms have a similar performance as was obtained when using 64 samples per window
%In \cref{fig:lineplot_gain_64}, we see that both methods enhance the output signal, achieving a positive gain over the spectrum, with the SSBT filter having a slightly better yield, for lower frequencies and the STFT filter leading to better results for higher frequencies. This result is reinforced by what was obtained in \cref{fig:heatmap_gain_64}.

\input{figures/io_input/aux_data_64.tex}

\begin{figure}[H]
	\centering
	\input{figures/64.lineplot_dsdi.tex}
	\caption{Window-average DSDI.}
	\label{fig:lineplot_dsdi_64}
\end{figure}

\begin{figure}[H]
	\centering
	\input{figures/64.lineplot_gain.tex}
	\caption{Window-average SNR gain.}
	\label{fig:lineplot_gain_64}
\end{figure}

\begin{figure}[H]
	\centering
	\input{figures/64.heatmap_gain.tex}
	
	\tikzsetnextfilename{heatmap_gain_64}%
	\vspace*{0.4em}
	\ref*{heatmap_SSBT_64}
	\caption{Per-window SNR gain.}
	\label{fig:heatmap_gain_64}
\end{figure}



%\subsection{Results - 128 samples/window}
%
%Here, we now use windows of 128 samples for the transforms.
%
%%For both transforms, the performance here is similar to that when using 64.
%
%\input{figures/io_input/aux_data_128.tex}
%\begin{figure}[H]
%	\centering
%	\input{figures/128.lineplot_gain.tex}
%	\caption{Window-average SNR gain.}
%	\label{fig:lineplot_gain_128}
%\end{figure}
%
%\begin{figure}[H]
%	\centering
%	\input{figures/128.heatmap_gain.tex}
%
%	\tikzsetnextfilename{heatmap_gain_128}%
%	\vspace*{0.4em}
%	\ref*{heatmap_SSBT_128}
%	\caption{Per-window SNR gain.}
%	\label{fig:heatmap_gain_128}
%\end{figure}
%
%\begin{figure}[H]
%\centering
%\input{figures/128.lineplot_dsdi.tex}
%\caption{Window-average DSDI.}
%\label{fig:lineplot_dsdi_128}
%\end{figure}