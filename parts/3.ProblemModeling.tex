\section{Problem modeling}

Let there be an uniform linear array with $M$ sensors, spaced $\delta_x$ apart. This array is in an environment where there are a desired and an interfering sources (respectively $x[n]$ and $v[n]$), and also uncorrelated noise at each sensor, $r_m[n]$.

\subsection{Time-domain basic signal model}
We denote $h_m[n]$ as the impulse response between the desired source and the $m$-th sensor, and similarly for $g_m[n]$ and the interfering source.

We can write the perceived signal $y_m[n]$ in the $m$-th sensor as
\begin{equation}\label{eq:pm:basic_model-1}
	y_m[n] = h_m[n] \ast x[n] + g_m[n] \ast v[n] + r_m[n]
\end{equation}

\subsection{Relative-impulse-response time-domain signal model}
Let $m'$ be the reference sensor's index, and without any compromise assume $m' = 1$. We then define $x_m[n] = h_m[n] \ast x[n]$ as the desired signal at the $m$-th sensor (resp. $v_m[n]$), and $b_m[n]$ such that
\begin{equation}
	b_m[n] \ast x_1[n] = h_m[n] \ast x[n]
\end{equation}
with $b_m[n]$ being the \emph{relative} impulse response between the reference and the $m$-th sensors for the desired signal (resp. $c_m[n]$ for the interfering signal). Now, we have that
\begin{equation}\label{eq:pm:basic_model-2}
	y_m[n] = b_m[n] \ast x_1[n] + c_m[n] \ast v_1[n] + r_m[n]
\end{equation}

We lastly can write $b_m[n]$ as the solution to the equation
\begin{eqnarray}
	b_m[n] \ast h_1[n] = h_m[n]
\end{eqnarray}

In the model from \cref{eq:pm:basic_model-1}, the system was causal. After rewriting it and arriving at \cref{eq:pm:basic_model-2}, it most likely is non-causal. As an example, if we assume the desired source to arrive lastly at the reference sensor, then $h_1[n]$ would be delayed compared to all other $h_m[n]$, and therefore $b_m[n]$ would need to have a forward (non-causal) contribution.

\subsection{Non-causal zero-padding}
\label{subsec:zero_padding}

There are some conditions that would be desired to better study this problem:
\begin{enumerate}
	\item All $b_m[n]$ are sufficiently recorded
	\item All $b_m[n]$ are the same length
	\item $b_1[n]$ starts at (what will be) the start of a window
\end{enumerate}

Assume that the worst-case scenario for the delay between the signal arriving at any sensor (except for the reference), and it arriving at the reference sensor, is $\Delta_t$ seconds. In samples,
\begin{equation}
\Delta_n = \ceil{\Delta_t f_s}
\end{equation}
with $f_s$ being the sampling frequency.

Therefore, $b_1[n]$ starts at most $\Delta_n$ samples after $b_M[n]$. We will assume the first condition is met by definition. If we let $\Theta$ be the overlap between the frequency-analysis windows, then to satisfy the third condition we need that
\begin{equation}
	\Delta_n^* = \ceil{\frac{\Delta_n}{\Theta}}\Theta
\end{equation}
is the number of zeroes needed at the start of $b_1[n]$ in order to guarantee the third condition. Since we assume there already were $\Delta_n$ in order to satisfy the first condition, then all $\b_m[n]$ need to be left-padded with $\Delta_n^* - \Delta_n$ samples.

Finally, to satisfy the second condition, we right-pad all $b_m[n]$ with enough zeroes such that they all have the same length as the longest one.

It is important to take into account the delay between sensors and the length of the window (in seconds). If the time it takes for the desired signal to go from the reference to the farthest sensor is longer than the length of the window, more than one window will need to be considered desired speech. Assuming that the farthest sensor is a distance $\delta$ from the reference, then this isn't a problem as long as $\frac{\delta}{c} < \frac{K}{f_s}$.

\subsection{Time-frequency-domain signal model}

In the time-frequency-domain (through a transform such as the STFT),
\begin{equation}
	Y_m[l,k] = B_m[l,k] \ast X_1[l,k] + C_m[l,k] \ast V_1[l,k] + R_m[l,k]
\end{equation}

Vectorizing the signals in the window axis (as defined in \cref{sec:filter_banks_ctf_model}), we have that

\begin{equation}
	Y_m[l,k] = \tr{\bvb{m}}[k] \bvx{1}[l,k] + \tr{\bvc{m}}[k] \bvv{1}[l,k] + R_m[l,k]
\end{equation}

Now stacking the signals in the sensor axis, we have that
\begin{equation}
	\bvy[l,k] = \tr{\bvB{}}[k] \bvx{1}[l,k] + \tr{\bvC{1}}[k] \bvv{1}[l,k] + \bvr[l,k]
\end{equation}
where
\begin{equation}
	\bvy[l,k] = \tr{ \tup{ {y_1[l,k]} , \cdots , {y_M[l,k]} } }
\end{equation}
and similarly for the other variables. In this situation, $\bvB{}[k]$ and $\bvC{}[k]$ are $\sz{L_H}{M}$ and $\sz{L_G}{M}$ matrices respectively, $\bvx{1}[l,k]$ and $\bvv{1}[l,k]$ are $\sz{L_H}{1}$ and $\sz{L_G}{1}$ vectors respectively, and $\bvy[l,k]$ and $\bvr[l,k]$ are $\sz{M}{1}$ vectors.

In this form, $\bvB{}[k]$ is a time-invariant variable equivalent to the impulse responses of the system for all sensors and (relevant) windows. However, it being a matrix complicates the analysis.

%\begin{method}*
%We let $\bvx[h]{}[l,k] = \frac{\bvx{}[l,k]}{x[l,k]}$, where $x[l,k]$ is the most-recent sample of the desired signal. With this, we define $\bvd{x}[l,k]$ as a (window-variant) transfer function for $x[l,k]$,
%\begin{equations}
%    \bvd{x}[l,k]
%    & = \tr{\bvH{}}[k] \bvx[h]{}[l,k] \\
%    & = \tr{\bvH{}}[k] \frac{\bvx{}[l,k]}{x[l,k]}
%\end{equations}
%and in a similar way we define $\bvd{v}[l,k]$. In this new framework, $\bvd{x}[l,k]$ and $\bvd{v}[l,k]$ are both $\sz{M}{1}$ vectors. With this,
%\begin{equation}
%    \bvy[l,k] = \bvd{x}[l,k] x[l,k] + \bvd{v}[l,k] v[l,k] + \bvr[l,k]
%\end{equation}
%and from here the study is the same as is usually done for beamforming, with $\bvd{x}[l,k]$ being the steering vector (considering reverberation) of $x[l,k]$, and the same for $\bvd{v}[l,k]$.
%
%We can also write
%\begin{equation}
%    \bvy[l,k] = \bvd{x}[l,k] x[l,k] + \bvw[l,k]
%\end{equation}
%with $\bvw[l,k]$ being the undesired signal (composed of undesired sources + noise)
%\begin{equation}
%    \bvw[l,k] = \bvd{v}[l,k] v[l,k] + \bvr[l,k]
%\end{equation}
%\end{method}
%%
\subsection{Reverb-aware reformulation}

We define $\Delta$ as
\begin{equations}
	\Delta 
	& = \ceil{\frac{\Delta_n}{\Theta}}
\end{equations}
as the window-index that starts when $b_1[n]$ starts (which happens given the pre-processing done in \cref{subsec:zero_padding}). With this, we write
\begin{equation}
	\tr{\bvB}[k] \bvx{1}[l,k] = \bvd{x}[k] X_1[l,k] + \sum_{\substack{l'=0 \\ l' \neq \Delta}}^{L_H-1} \bvp{B,l'}[k] X_1[l-l',k]
\end{equation}
where
\begin{subgather}
	\bvd{x}[k] = \tr{ \tup{ \bvB{[\Delta,0]}[k] , \cdots , \bvB{[\Delta,M-1]}[k] } } \\
	\bvp{B,l'}[k] = \tr{ \tup{ \bvB{[l',0]}[k] , \cdots , \bvB{[l',M-1]}[k] } } 
\end{subgather}
and $X_1[l-l',k]$ is $l'$-th sample of $\bvx{1}[l,k]$. With this, $\bvd{x}[k] X_1[l,k]$ is the desired speech component of $\tr{\bvB}[k] \bvx{1}[l,k]$, and the summation over $l'$ is an undesired speech component. That is, we assume that all windows for the desired signal besides the first one are noise, and are to be canceled (or minimized). We define $\bvp{C,l''}[k]$ similarly but for $\bvv{1}[l,k]$, such that
\begin{equation}
	\tr{\bvC}[k] \bvv{1}[l,k] = \sum_{l''=0}^{L_G-1} \bvp{C,l''}[k] V_1[l-l'',k]
\end{equation}

Therefore, we can write
\begin{equation}
	\bvy[l,k] = \bvd{x}[k] X_1[l,k] + \sum_{\substack{l'=0 \\ l' \neq \Delta}}^{L_H-1} \bvp{B,l'}[k] X_1[l-l',k] + \sum_{l''=0}^{L_G-1} \bvp{C,l''}[k] V_1[l-l'',k] + \bvr[l,k]
\end{equation}
or also
\begin{equation}
	\bvy[l,k] = \bvd{x}[k] X_1[l,k] + \bvw[l,k]
\end{equation}
with $\bvw[l,k]$ being the undesired signal (undesired speech components $+$ interfering source $+$ noise), given by
\begin{equation}\label{eq:bvw_sum_components}
	\bvw[l,k] = \sum_{\substack{l'=0 \\ l' \neq \Delta}}^{L_H-1} \bvp{B,l'}[k] X_1[l-l',k] + \sum_{l''=0}^{L_G-1} \bvp{C,l''}[k] V_1[l-l'',k] + \bvr[l,k]
\end{equation}

\subsection{Filtering and beamforming}

We use an LTI filter $\bvf[l,k]$, in such a way that
\begin{equations}
	z[l,k]
	& = \he{\bvf}[l,k] \bvy[l,k] \\
	& \approx x[l,k]
\end{equations}

In this sense, the distortionless constraint is
\begin{equation}
	\he{\bvf}[l,k] \bvd{x}[l,k] = 1
\end{equation}

% From now on, the $[l,k]$ indexes will be omitted for clarity.

We will use a MVDR beamformer, which minimizes the noise under the distortionless constraint, whose solution is written as
\begin{equation}
	\label{eq:sec3:mvdr_beamformer}
	\bvf{\mvdr}[l,k] = \frac{\inv{\Corr{\bvw}}[l,k] \bvd{x}[k]}{\he{\bvd{x}}[k] \inv{\Corr{\bvw}}[l,k] \bvd{x}[k]}
\end{equation}

In this situation, through \cref{eq:bvw_sum_components} we can write $\Corr{\bvw}[l,k]$ as
\begin{equations}
	\Corr{\bvw}[l,k]
	& = \sum_{\substack{l'=0 \\ l' \neq \Delta}}^{L_H-1} \he{\bvp{B,l'}}[k] \bvp{B,l'}[k] \var{X}[l-l',k] \\
	& + \sum_{l''=0}^{L_G-1} \he{\bvp{C,l''}}[k] \bvp{C,l''}[k] \var{V}[l-l'',k] \\
	& + \id{M} \var{R}[l,k]
\end{equations}
where $\id{M}$ is the $\sz{M}{M}$ identity matrix, and we assume that the distribution of the noise $\bvr[l,k]$ is the same for all sensors (that is, its variance is the same for all sensors).

%If we assume that the distribution of $\var{X}$ is time-invariant (that is, it is stationary), then $\var{X}[l,k] = \var{X}[l',k] \defas \var{X}[k]$. The same is applied to achieve $\var{V}[k]$ and $\var{R}[k]$. With this assumption, we have that $\Corr{\bvw}[l,k] \defas \Corr{\bvw}[k]$ and $\bvf{\mvdr}[l,k] \defas \bvf{\mvdr}[k]$ are time-invariant as well, and therefore the beamformer's coefficients depend only on the bin index.
%
%this assumption of time-invariancy on the distribution of the relevant signals is useful to simplify the modeling. Although over long periods of time their variances may change, in short periods it stays somewhat constant.