\section{Signal modeling}
\label{sec1:signal_model}
Let there be a system defined as
\begin{equations}
	y[n]
	& = x[n] \ast h[n] + v[n] \\
	& = d[n] + v[n]
\end{equations}

Passing through a short-time transform (STFT or SSBT), it is given in the time-frequency domain as
\begin{equations}
	\dy[l,k]
	& = \dd[l,k] + \dv[l,k] \\
	& = \sum_{k'=0}^{N-1} \sum_{l'=0}^{L_h-1} \dx[l-l',k'] \dh[l',\{k,k'\}] + \dv[l,k]
\end{equations}
in which:\\
$\dy[l,k]$ is the transform of $y[n]$, at window-index (or downsampled time) $l$ and bin $k$ (and the same for $\dd[l,k]$, $\dx[l-l',k]$ and $\dv[l,k]$)\\
$\dh[l', \{k,k'\}]$ is the frequency response at time $l'$ between bin $k$ and bin $k'$\\
$N$ is the number of frequency bins\\
$L_h$ is the length of $\dh[l',\{k,k'\}]$ in the sample axis.

The second summation (over $l'$) can be seen as a time-convolution between $\dx[l,k]$ and $\dh[l,\{k,k'\}]$, and the first summation (over $k'$) as either a frequency-convolution, or as an interference of the $k'$-th bin of $\dx[l,k]$ and the $k$-th bin of $\dy[l,k]$. We call $\dh[l, \{k,k\}]$ as the band-to-band response (or band-to-band filter), and $\dh[l,\{k,k'\}]$ for $k \neq k'$ as the crossband response.

We denote $\bvh[d][k,k']$ as the time-stacking vector of length $L_h$ defined as
\begin{equation}
	\bvh[d][k,k'] = \tr{ \tup{ {\dh[0,\{k,k'\}]} , {\dh[1,\{k,k'\}]} , , {\dh[L_h-1,\{k,k'\}]} } }
\end{equation}
and $\bvh{}[k]$ as the stacking of $\bvh[d][k,k']$ for $k' \in \{0,~N-1\}$:
\begin{equation}
	\bvh{}[k] = \tr{ \tup{ {\tr{\bvh[d]}[k,0]} , {\tr{\bvh[d]}[k,1]} ,, {\tr{\bvh[d]}[k,N-1]} } }
\end{equation}
with $\bvh{}[k]$ being a $\sz{NL_h}{1}$ vector.

By writing
\begin{subgather}
	\bvx[d][l,k] = \tr{ \tup{ {\dx[l,k]} , {\dx[l-1,k]} ,, {\dx[l-L_h+1, k]} } } \\
	\bvx{}[l] = \tr{ \tup{ {\tr{\bvx[d]}[l,0]} , {\tr{\bvx[d]}[l,1]} ,, {\tr{\bvx[d]}[l,N-1]} } }
\end{subgather}
then
\begin{equation}
	\dy[l,k] = \tr{\bvx{}[l]} \bvh{}[k] + \dv{}[l,k]
\end{equation}

If we write
\begin{equation}
	\bvX = \tr{ \tup{ {\bvx}[0] , {\bvx}[1] ,, {\bvx}[L_y-1] } }
\end{equation}
then
\begin{equation}
	\bvy[k] = \bvX \bvh{}[k] + \bvv[k]
\end{equation}
where
\begin{equation}
	\bvy[k] = \tr{ \tup{ {\dy[0,k]} , {\dy[1,k]} ,, {\dy[L_y-1,k]} } }
\end{equation}
with $L_y = L_h + L_x - 1$ being the output signal's length. For such, we assume that $\bvx[d][l,k] = 0$ for $l < 0$.

We have that $\bvx[d][l,k]$ is a $\sz{L_h}{1}$ vector, $\bvx{}[l]$ is a $\sz{NL_h}{1}$ vector, and $\bvX$ is a $\sz{L_y}{NL_h}$ matrix. $\bvh{}[k]$ is a $\sz{NL_h}{1}$ vector. $\bvy[k]$ and $\bvv[k]$ are $\sz{L_y}{1}$ vectors.